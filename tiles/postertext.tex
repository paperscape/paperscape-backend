\documentclass[11pt]{article}

% this gives a nice font
\usepackage{charter}

% remove space from top of page
\setlength{\voffset}{-0.9in}
\setlength{\topmargin}{0pt}
\setlength{\headheight}{0pt}
\setlength{\headsep}{0pt}
\addtolength{\textwidth}{1in}

% remove space from left of page
\setlength{\hoffset}{-0.9in}
\setlength{\oddsidemargin}{0pt}
\addtolength{\textheight}{1in}

% set paragraph style
\setlength{\parindent}{0pt}
\setlength{\parskip}{0.6cm plus4mm minus3mm}

\begin{document}

{\Huge\bf Paperscape.org}

{\large A project by {\bf Damien P. George} and {\bf Rob Knegjens}}

%Paperscape is a tool to visualise the arXiv, an open, online repository for scientific research papers. The Paperscape map currently includes all (non-withdrawn) papers from the arXiv and is updated daily.

Paperscape is an interactive map that visualises the arXiv, an open, online repository for scientific research papers. 
The map, which can be explored online by panning and zooming, currently includes all 880,000+ papers from the arXiv and is updated daily.

%Each paper in the map is represented by a circle, with the area of the circle proportional to the number of citations that paper has. 
%In laying out the map, an N-body algorithm is run to determine positions based on references between the papers. 
%There are two “forces” involved in the N-body calculation: each paper is repelled from all other papers using an anti-gravity inverse-distance force, and each paper is attracted to all of its references using a spring modelled by Hooke’s law. 
%We further demand that there is no overlap of the papers.

Each scientific paper is represented in the map by a circle whose size is determined by the number of times that paper has been cited by others.
A paper's position in the map is determined by both its citation links (papers that cite it) and its reference links (papers it refers to).
These links pull related papers together, whereas papers with no or few links in common push each other away.

%The map is rendered simply as a solid circle for each paper. The colour of the circle denotes the arXiv category of the paper, and the brightness indicates age. Brightness is sometime difficult to discern, and we are working on adding a heat-map overlay to indicate clearly the areas of the map which have the most recent activity.

%As you zoom in on the map labels will start to appear on individual papers. 
%These labels are (mostly) automatically extracted by analysing word frequency in the title and abstract of the paper, and are generally indicative of the subject matter of that paper. 
%Zooming in closer also shows the author(s) of the paper. 
%If a paper is deemed to be a review paper, or a set of lectures, this is noted.

Papers are coloured according to their scientific category.
As a result coloured "continents" are seen to emerge, such as theoretical high energy physics (blue) or astrophysics (pink).
At their interface one finds cross-disciplinary fields, such as dark matter and cosmological inflation.
Looking within a continent reveals substructures representing more specific fields of research.
%The automatically extracted keywords that appear on top of papers help to identify interesting papers and fields.

References (and citation counting) are extracted by processing the TeX/LaTeX and PDF source obtained from the arXiv. 
%This is done automatically each morning, and the map is finished updating about 3 or 4 hours after the arXiv’s new listing is announced. 
Some categories (noticeably hep-th and hep-ph) have better reference extraction than others and so the map for these areas has more variation in paper size and more structure. 
We are working on improving the reference extraction.

\end{document}
